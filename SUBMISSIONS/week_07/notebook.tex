
% Default to the notebook output style

    


% Inherit from the specified cell style.




    
\documentclass[11pt]{article}

    
    
    \usepackage[T1]{fontenc}
    % Nicer default font (+ math font) than Computer Modern for most use cases
    \usepackage{mathpazo}

    % Basic figure setup, for now with no caption control since it's done
    % automatically by Pandoc (which extracts ![](path) syntax from Markdown).
    \usepackage{graphicx}
    % We will generate all images so they have a width \maxwidth. This means
    % that they will get their normal width if they fit onto the page, but
    % are scaled down if they would overflow the margins.
    \makeatletter
    \def\maxwidth{\ifdim\Gin@nat@width>\linewidth\linewidth
    \else\Gin@nat@width\fi}
    \makeatother
    \let\Oldincludegraphics\includegraphics
    % Set max figure width to be 80% of text width, for now hardcoded.
    \renewcommand{\includegraphics}[1]{\Oldincludegraphics[width=.8\maxwidth]{#1}}
    % Ensure that by default, figures have no caption (until we provide a
    % proper Figure object with a Caption API and a way to capture that
    % in the conversion process - todo).
    \usepackage{caption}
    \DeclareCaptionLabelFormat{nolabel}{}
    \captionsetup{labelformat=nolabel}

    \usepackage{adjustbox} % Used to constrain images to a maximum size 
    \usepackage{xcolor} % Allow colors to be defined
    \usepackage{enumerate} % Needed for markdown enumerations to work
    \usepackage{geometry} % Used to adjust the document margins
    \usepackage{amsmath} % Equations
    \usepackage{amssymb} % Equations
    \usepackage{textcomp} % defines textquotesingle
    % Hack from http://tex.stackexchange.com/a/47451/13684:
    \AtBeginDocument{%
        \def\PYZsq{\textquotesingle}% Upright quotes in Pygmentized code
    }
    \usepackage{upquote} % Upright quotes for verbatim code
    \usepackage{eurosym} % defines \euro
    \usepackage[mathletters]{ucs} % Extended unicode (utf-8) support
    \usepackage[utf8x]{inputenc} % Allow utf-8 characters in the tex document
    \usepackage{fancyvrb} % verbatim replacement that allows latex
    \usepackage{grffile} % extends the file name processing of package graphics 
                         % to support a larger range 
    % The hyperref package gives us a pdf with properly built
    % internal navigation ('pdf bookmarks' for the table of contents,
    % internal cross-reference links, web links for URLs, etc.)
    \usepackage{hyperref}
    \usepackage{longtable} % longtable support required by pandoc >1.10
    \usepackage{booktabs}  % table support for pandoc > 1.12.2
    \usepackage[inline]{enumitem} % IRkernel/repr support (it uses the enumerate* environment)
    \usepackage[normalem]{ulem} % ulem is needed to support strikethroughs (\sout)
                                % normalem makes italics be italics, not underlines
    

    
    
    % Colors for the hyperref package
    \definecolor{urlcolor}{rgb}{0,.145,.698}
    \definecolor{linkcolor}{rgb}{.71,0.21,0.01}
    \definecolor{citecolor}{rgb}{.12,.54,.11}

    % ANSI colors
    \definecolor{ansi-black}{HTML}{3E424D}
    \definecolor{ansi-black-intense}{HTML}{282C36}
    \definecolor{ansi-red}{HTML}{E75C58}
    \definecolor{ansi-red-intense}{HTML}{B22B31}
    \definecolor{ansi-green}{HTML}{00A250}
    \definecolor{ansi-green-intense}{HTML}{007427}
    \definecolor{ansi-yellow}{HTML}{DDB62B}
    \definecolor{ansi-yellow-intense}{HTML}{B27D12}
    \definecolor{ansi-blue}{HTML}{208FFB}
    \definecolor{ansi-blue-intense}{HTML}{0065CA}
    \definecolor{ansi-magenta}{HTML}{D160C4}
    \definecolor{ansi-magenta-intense}{HTML}{A03196}
    \definecolor{ansi-cyan}{HTML}{60C6C8}
    \definecolor{ansi-cyan-intense}{HTML}{258F8F}
    \definecolor{ansi-white}{HTML}{C5C1B4}
    \definecolor{ansi-white-intense}{HTML}{A1A6B2}

    % commands and environments needed by pandoc snippets
    % extracted from the output of `pandoc -s`
    \providecommand{\tightlist}{%
      \setlength{\itemsep}{0pt}\setlength{\parskip}{0pt}}
    \DefineVerbatimEnvironment{Highlighting}{Verbatim}{commandchars=\\\{\}}
    % Add ',fontsize=\small' for more characters per line
    \newenvironment{Shaded}{}{}
    \newcommand{\KeywordTok}[1]{\textcolor[rgb]{0.00,0.44,0.13}{\textbf{{#1}}}}
    \newcommand{\DataTypeTok}[1]{\textcolor[rgb]{0.56,0.13,0.00}{{#1}}}
    \newcommand{\DecValTok}[1]{\textcolor[rgb]{0.25,0.63,0.44}{{#1}}}
    \newcommand{\BaseNTok}[1]{\textcolor[rgb]{0.25,0.63,0.44}{{#1}}}
    \newcommand{\FloatTok}[1]{\textcolor[rgb]{0.25,0.63,0.44}{{#1}}}
    \newcommand{\CharTok}[1]{\textcolor[rgb]{0.25,0.44,0.63}{{#1}}}
    \newcommand{\StringTok}[1]{\textcolor[rgb]{0.25,0.44,0.63}{{#1}}}
    \newcommand{\CommentTok}[1]{\textcolor[rgb]{0.38,0.63,0.69}{\textit{{#1}}}}
    \newcommand{\OtherTok}[1]{\textcolor[rgb]{0.00,0.44,0.13}{{#1}}}
    \newcommand{\AlertTok}[1]{\textcolor[rgb]{1.00,0.00,0.00}{\textbf{{#1}}}}
    \newcommand{\FunctionTok}[1]{\textcolor[rgb]{0.02,0.16,0.49}{{#1}}}
    \newcommand{\RegionMarkerTok}[1]{{#1}}
    \newcommand{\ErrorTok}[1]{\textcolor[rgb]{1.00,0.00,0.00}{\textbf{{#1}}}}
    \newcommand{\NormalTok}[1]{{#1}}
    
    % Additional commands for more recent versions of Pandoc
    \newcommand{\ConstantTok}[1]{\textcolor[rgb]{0.53,0.00,0.00}{{#1}}}
    \newcommand{\SpecialCharTok}[1]{\textcolor[rgb]{0.25,0.44,0.63}{{#1}}}
    \newcommand{\VerbatimStringTok}[1]{\textcolor[rgb]{0.25,0.44,0.63}{{#1}}}
    \newcommand{\SpecialStringTok}[1]{\textcolor[rgb]{0.73,0.40,0.53}{{#1}}}
    \newcommand{\ImportTok}[1]{{#1}}
    \newcommand{\DocumentationTok}[1]{\textcolor[rgb]{0.73,0.13,0.13}{\textit{{#1}}}}
    \newcommand{\AnnotationTok}[1]{\textcolor[rgb]{0.38,0.63,0.69}{\textbf{\textit{{#1}}}}}
    \newcommand{\CommentVarTok}[1]{\textcolor[rgb]{0.38,0.63,0.69}{\textbf{\textit{{#1}}}}}
    \newcommand{\VariableTok}[1]{\textcolor[rgb]{0.10,0.09,0.49}{{#1}}}
    \newcommand{\ControlFlowTok}[1]{\textcolor[rgb]{0.00,0.44,0.13}{\textbf{{#1}}}}
    \newcommand{\OperatorTok}[1]{\textcolor[rgb]{0.40,0.40,0.40}{{#1}}}
    \newcommand{\BuiltInTok}[1]{{#1}}
    \newcommand{\ExtensionTok}[1]{{#1}}
    \newcommand{\PreprocessorTok}[1]{\textcolor[rgb]{0.74,0.48,0.00}{{#1}}}
    \newcommand{\AttributeTok}[1]{\textcolor[rgb]{0.49,0.56,0.16}{{#1}}}
    \newcommand{\InformationTok}[1]{\textcolor[rgb]{0.38,0.63,0.69}{\textbf{\textit{{#1}}}}}
    \newcommand{\WarningTok}[1]{\textcolor[rgb]{0.38,0.63,0.69}{\textbf{\textit{{#1}}}}}
    
    
    % Define a nice break command that doesn't care if a line doesn't already
    % exist.
    \def\br{\hspace*{\fill} \\* }
    % Math Jax compatability definitions
    \def\gt{>}
    \def\lt{<}
    % Document parameters
    \title{Week\_07\_HW}
    
    
    

    % Pygments definitions
    
\makeatletter
\def\PY@reset{\let\PY@it=\relax \let\PY@bf=\relax%
    \let\PY@ul=\relax \let\PY@tc=\relax%
    \let\PY@bc=\relax \let\PY@ff=\relax}
\def\PY@tok#1{\csname PY@tok@#1\endcsname}
\def\PY@toks#1+{\ifx\relax#1\empty\else%
    \PY@tok{#1}\expandafter\PY@toks\fi}
\def\PY@do#1{\PY@bc{\PY@tc{\PY@ul{%
    \PY@it{\PY@bf{\PY@ff{#1}}}}}}}
\def\PY#1#2{\PY@reset\PY@toks#1+\relax+\PY@do{#2}}

\expandafter\def\csname PY@tok@w\endcsname{\def\PY@tc##1{\textcolor[rgb]{0.73,0.73,0.73}{##1}}}
\expandafter\def\csname PY@tok@c\endcsname{\let\PY@it=\textit\def\PY@tc##1{\textcolor[rgb]{0.25,0.50,0.50}{##1}}}
\expandafter\def\csname PY@tok@cp\endcsname{\def\PY@tc##1{\textcolor[rgb]{0.74,0.48,0.00}{##1}}}
\expandafter\def\csname PY@tok@k\endcsname{\let\PY@bf=\textbf\def\PY@tc##1{\textcolor[rgb]{0.00,0.50,0.00}{##1}}}
\expandafter\def\csname PY@tok@kp\endcsname{\def\PY@tc##1{\textcolor[rgb]{0.00,0.50,0.00}{##1}}}
\expandafter\def\csname PY@tok@kt\endcsname{\def\PY@tc##1{\textcolor[rgb]{0.69,0.00,0.25}{##1}}}
\expandafter\def\csname PY@tok@o\endcsname{\def\PY@tc##1{\textcolor[rgb]{0.40,0.40,0.40}{##1}}}
\expandafter\def\csname PY@tok@ow\endcsname{\let\PY@bf=\textbf\def\PY@tc##1{\textcolor[rgb]{0.67,0.13,1.00}{##1}}}
\expandafter\def\csname PY@tok@nb\endcsname{\def\PY@tc##1{\textcolor[rgb]{0.00,0.50,0.00}{##1}}}
\expandafter\def\csname PY@tok@nf\endcsname{\def\PY@tc##1{\textcolor[rgb]{0.00,0.00,1.00}{##1}}}
\expandafter\def\csname PY@tok@nc\endcsname{\let\PY@bf=\textbf\def\PY@tc##1{\textcolor[rgb]{0.00,0.00,1.00}{##1}}}
\expandafter\def\csname PY@tok@nn\endcsname{\let\PY@bf=\textbf\def\PY@tc##1{\textcolor[rgb]{0.00,0.00,1.00}{##1}}}
\expandafter\def\csname PY@tok@ne\endcsname{\let\PY@bf=\textbf\def\PY@tc##1{\textcolor[rgb]{0.82,0.25,0.23}{##1}}}
\expandafter\def\csname PY@tok@nv\endcsname{\def\PY@tc##1{\textcolor[rgb]{0.10,0.09,0.49}{##1}}}
\expandafter\def\csname PY@tok@no\endcsname{\def\PY@tc##1{\textcolor[rgb]{0.53,0.00,0.00}{##1}}}
\expandafter\def\csname PY@tok@nl\endcsname{\def\PY@tc##1{\textcolor[rgb]{0.63,0.63,0.00}{##1}}}
\expandafter\def\csname PY@tok@ni\endcsname{\let\PY@bf=\textbf\def\PY@tc##1{\textcolor[rgb]{0.60,0.60,0.60}{##1}}}
\expandafter\def\csname PY@tok@na\endcsname{\def\PY@tc##1{\textcolor[rgb]{0.49,0.56,0.16}{##1}}}
\expandafter\def\csname PY@tok@nt\endcsname{\let\PY@bf=\textbf\def\PY@tc##1{\textcolor[rgb]{0.00,0.50,0.00}{##1}}}
\expandafter\def\csname PY@tok@nd\endcsname{\def\PY@tc##1{\textcolor[rgb]{0.67,0.13,1.00}{##1}}}
\expandafter\def\csname PY@tok@s\endcsname{\def\PY@tc##1{\textcolor[rgb]{0.73,0.13,0.13}{##1}}}
\expandafter\def\csname PY@tok@sd\endcsname{\let\PY@it=\textit\def\PY@tc##1{\textcolor[rgb]{0.73,0.13,0.13}{##1}}}
\expandafter\def\csname PY@tok@si\endcsname{\let\PY@bf=\textbf\def\PY@tc##1{\textcolor[rgb]{0.73,0.40,0.53}{##1}}}
\expandafter\def\csname PY@tok@se\endcsname{\let\PY@bf=\textbf\def\PY@tc##1{\textcolor[rgb]{0.73,0.40,0.13}{##1}}}
\expandafter\def\csname PY@tok@sr\endcsname{\def\PY@tc##1{\textcolor[rgb]{0.73,0.40,0.53}{##1}}}
\expandafter\def\csname PY@tok@ss\endcsname{\def\PY@tc##1{\textcolor[rgb]{0.10,0.09,0.49}{##1}}}
\expandafter\def\csname PY@tok@sx\endcsname{\def\PY@tc##1{\textcolor[rgb]{0.00,0.50,0.00}{##1}}}
\expandafter\def\csname PY@tok@m\endcsname{\def\PY@tc##1{\textcolor[rgb]{0.40,0.40,0.40}{##1}}}
\expandafter\def\csname PY@tok@gh\endcsname{\let\PY@bf=\textbf\def\PY@tc##1{\textcolor[rgb]{0.00,0.00,0.50}{##1}}}
\expandafter\def\csname PY@tok@gu\endcsname{\let\PY@bf=\textbf\def\PY@tc##1{\textcolor[rgb]{0.50,0.00,0.50}{##1}}}
\expandafter\def\csname PY@tok@gd\endcsname{\def\PY@tc##1{\textcolor[rgb]{0.63,0.00,0.00}{##1}}}
\expandafter\def\csname PY@tok@gi\endcsname{\def\PY@tc##1{\textcolor[rgb]{0.00,0.63,0.00}{##1}}}
\expandafter\def\csname PY@tok@gr\endcsname{\def\PY@tc##1{\textcolor[rgb]{1.00,0.00,0.00}{##1}}}
\expandafter\def\csname PY@tok@ge\endcsname{\let\PY@it=\textit}
\expandafter\def\csname PY@tok@gs\endcsname{\let\PY@bf=\textbf}
\expandafter\def\csname PY@tok@gp\endcsname{\let\PY@bf=\textbf\def\PY@tc##1{\textcolor[rgb]{0.00,0.00,0.50}{##1}}}
\expandafter\def\csname PY@tok@go\endcsname{\def\PY@tc##1{\textcolor[rgb]{0.53,0.53,0.53}{##1}}}
\expandafter\def\csname PY@tok@gt\endcsname{\def\PY@tc##1{\textcolor[rgb]{0.00,0.27,0.87}{##1}}}
\expandafter\def\csname PY@tok@err\endcsname{\def\PY@bc##1{\setlength{\fboxsep}{0pt}\fcolorbox[rgb]{1.00,0.00,0.00}{1,1,1}{\strut ##1}}}
\expandafter\def\csname PY@tok@kc\endcsname{\let\PY@bf=\textbf\def\PY@tc##1{\textcolor[rgb]{0.00,0.50,0.00}{##1}}}
\expandafter\def\csname PY@tok@kd\endcsname{\let\PY@bf=\textbf\def\PY@tc##1{\textcolor[rgb]{0.00,0.50,0.00}{##1}}}
\expandafter\def\csname PY@tok@kn\endcsname{\let\PY@bf=\textbf\def\PY@tc##1{\textcolor[rgb]{0.00,0.50,0.00}{##1}}}
\expandafter\def\csname PY@tok@kr\endcsname{\let\PY@bf=\textbf\def\PY@tc##1{\textcolor[rgb]{0.00,0.50,0.00}{##1}}}
\expandafter\def\csname PY@tok@bp\endcsname{\def\PY@tc##1{\textcolor[rgb]{0.00,0.50,0.00}{##1}}}
\expandafter\def\csname PY@tok@fm\endcsname{\def\PY@tc##1{\textcolor[rgb]{0.00,0.00,1.00}{##1}}}
\expandafter\def\csname PY@tok@vc\endcsname{\def\PY@tc##1{\textcolor[rgb]{0.10,0.09,0.49}{##1}}}
\expandafter\def\csname PY@tok@vg\endcsname{\def\PY@tc##1{\textcolor[rgb]{0.10,0.09,0.49}{##1}}}
\expandafter\def\csname PY@tok@vi\endcsname{\def\PY@tc##1{\textcolor[rgb]{0.10,0.09,0.49}{##1}}}
\expandafter\def\csname PY@tok@vm\endcsname{\def\PY@tc##1{\textcolor[rgb]{0.10,0.09,0.49}{##1}}}
\expandafter\def\csname PY@tok@sa\endcsname{\def\PY@tc##1{\textcolor[rgb]{0.73,0.13,0.13}{##1}}}
\expandafter\def\csname PY@tok@sb\endcsname{\def\PY@tc##1{\textcolor[rgb]{0.73,0.13,0.13}{##1}}}
\expandafter\def\csname PY@tok@sc\endcsname{\def\PY@tc##1{\textcolor[rgb]{0.73,0.13,0.13}{##1}}}
\expandafter\def\csname PY@tok@dl\endcsname{\def\PY@tc##1{\textcolor[rgb]{0.73,0.13,0.13}{##1}}}
\expandafter\def\csname PY@tok@s2\endcsname{\def\PY@tc##1{\textcolor[rgb]{0.73,0.13,0.13}{##1}}}
\expandafter\def\csname PY@tok@sh\endcsname{\def\PY@tc##1{\textcolor[rgb]{0.73,0.13,0.13}{##1}}}
\expandafter\def\csname PY@tok@s1\endcsname{\def\PY@tc##1{\textcolor[rgb]{0.73,0.13,0.13}{##1}}}
\expandafter\def\csname PY@tok@mb\endcsname{\def\PY@tc##1{\textcolor[rgb]{0.40,0.40,0.40}{##1}}}
\expandafter\def\csname PY@tok@mf\endcsname{\def\PY@tc##1{\textcolor[rgb]{0.40,0.40,0.40}{##1}}}
\expandafter\def\csname PY@tok@mh\endcsname{\def\PY@tc##1{\textcolor[rgb]{0.40,0.40,0.40}{##1}}}
\expandafter\def\csname PY@tok@mi\endcsname{\def\PY@tc##1{\textcolor[rgb]{0.40,0.40,0.40}{##1}}}
\expandafter\def\csname PY@tok@il\endcsname{\def\PY@tc##1{\textcolor[rgb]{0.40,0.40,0.40}{##1}}}
\expandafter\def\csname PY@tok@mo\endcsname{\def\PY@tc##1{\textcolor[rgb]{0.40,0.40,0.40}{##1}}}
\expandafter\def\csname PY@tok@ch\endcsname{\let\PY@it=\textit\def\PY@tc##1{\textcolor[rgb]{0.25,0.50,0.50}{##1}}}
\expandafter\def\csname PY@tok@cm\endcsname{\let\PY@it=\textit\def\PY@tc##1{\textcolor[rgb]{0.25,0.50,0.50}{##1}}}
\expandafter\def\csname PY@tok@cpf\endcsname{\let\PY@it=\textit\def\PY@tc##1{\textcolor[rgb]{0.25,0.50,0.50}{##1}}}
\expandafter\def\csname PY@tok@c1\endcsname{\let\PY@it=\textit\def\PY@tc##1{\textcolor[rgb]{0.25,0.50,0.50}{##1}}}
\expandafter\def\csname PY@tok@cs\endcsname{\let\PY@it=\textit\def\PY@tc##1{\textcolor[rgb]{0.25,0.50,0.50}{##1}}}

\def\PYZbs{\char`\\}
\def\PYZus{\char`\_}
\def\PYZob{\char`\{}
\def\PYZcb{\char`\}}
\def\PYZca{\char`\^}
\def\PYZam{\char`\&}
\def\PYZlt{\char`\<}
\def\PYZgt{\char`\>}
\def\PYZsh{\char`\#}
\def\PYZpc{\char`\%}
\def\PYZdl{\char`\$}
\def\PYZhy{\char`\-}
\def\PYZsq{\char`\'}
\def\PYZdq{\char`\"}
\def\PYZti{\char`\~}
% for compatibility with earlier versions
\def\PYZat{@}
\def\PYZlb{[}
\def\PYZrb{]}
\makeatother


    % Exact colors from NB
    \definecolor{incolor}{rgb}{0.0, 0.0, 0.5}
    \definecolor{outcolor}{rgb}{0.545, 0.0, 0.0}



    
    % Prevent overflowing lines due to hard-to-break entities
    \sloppy 
    % Setup hyperref package
    \hypersetup{
      breaklinks=true,  % so long urls are correctly broken across lines
      colorlinks=true,
      urlcolor=urlcolor,
      linkcolor=linkcolor,
      citecolor=citecolor,
      }
    % Slightly bigger margins than the latex defaults
    
    \geometry{verbose,tmargin=1in,bmargin=1in,lmargin=1in,rmargin=1in}
    
    

    \begin{document}
    
    
    \maketitle
    
    

    
    \subsection{Week 7 Assignment - W200 Python for Data Science, UC
Berkeley
MIDS}\label{week-7-assignment---w200-python-for-data-science-uc-berkeley-mids}

Write code in this Jupyter Notebook to solve each of the following
problems. Each problem should have its solution in a separate cell.
Please upload this \textbf{Notebook} with your solutions to your GitHub
repository in your SUBMISSIONS/week\_07 folder by 11:59PM PST the night
before class the sync week 9 class.

    \subsection{Objectives:}\label{objectives}

\begin{itemize}
\tightlist
\item
  Demonstrate how to define classes
\item
  Design and implement class objects and class interactions
\item
  Understand how to call methods from both inside and outside of classes
\item
  Understand how to set internal attribute within a class
\end{itemize}

    \subsection{General Guidelines:}\label{general-guidelines}

\begin{itemize}
\tightlist
\item
  All calculations need to be done in the classes (that includes any
  formatting of the output)
\item
  Name your classes exactly as written in the problem statement
\item
  Do NOT make separate input() statements. The classes will be passed
  the input as shown in the examples
\item
  The examples given are samples of how we will test/grade your code.
  Please ensure your classes output the same information
\item
  Answer format is graded - please match the examples
\item
  User / function inputs do need to be validated or checked. (For
  example, if the problem states input an integer we will check it by
  inputting an integer)
\item
  Comments in your code are strongly suggested but won't be graded
\item
  This homework is mostly auto-graded. The blank code blocks are the
  auto-grading scripts - please do not delete these!
\item
  Your code needs to be written in the \#Your Code Here blocks or it
  wont be graded correctly.
\end{itemize}

    \subsection{Project Proposal}\label{project-proposal}

\textbf{Reminder!} Please complete your project proposal, as discussed
in class and outlined in the project\_1 folder. You may submit your 1-2
page proposal in a Google Doc or PDF. Please store your proposal (or a
link to it) in your repo under the project\_1 folder. This is due by
11:59PM PST the night before class the sync week 8 class! (not when this
homework is due!)

This is worth 10 points of your \textbf{project} grade (not the grade
for this homework).

    \subsection{7-1. A Quick Reading}\label{a-quick-reading}

Please read the following article and write a couple sentences of
reaction. What is the most interesting part?

Write code that is easy to delete, not easy to extend

This article is to explain an "architectural" perspective towards
thinking about writing in large code bases. This might not really apply
to the work that you are doing now but should provide some food for
thought on upcoming projects. Think about the author's perspective and
why he seems to have come to it. Please don't worry about knowing all
the terminology or programs/systems that he refers to. We want you to
extract what he's trying to say about writing code rather than the
intricacies of the low-level systems that is referring to.

    YOUR ANSWER HERE

    \begin{center}\rule{0.5\linewidth}{\linethickness}\end{center}

\section{Please do 2 out of the 3 parts
below.}\label{please-do-2-out-of-the-3-parts-below.}

\section{That is, please do two parts of Parts 7-2, 7-3 or
7-4.}\label{that-is-please-do-two-parts-of-parts-7-2-7-3-or-7-4.}

If you want to do all three parts please write a comment on which two
parts to grade. If a comment isn't found Parts 7-2 \& 7-3 will be
graded.

    \subsection{7-2. Deck of Cards}\label{deck-of-cards}

Please design two classes in this notebook as follows:

1. Please create a class called \textbf{PlayingCard}. This class should
have: - An attribute, "rank" that takes a value of 2-10, J, Q, K, or A.
- An attribute, "suit" that takes a value of "♠" "♥" "♦" or "♣". (If you
don't know how to make these characters you can cut and paste from this
block)\\
- An \textbf{init} function

2. Please create a class called \textbf{Deck}. This class should have: -
An attribute, "cards", that holds a list of PlayingCard objects. - An
\textbf{init} function that:

\begin{verbatim}
- By default stores a full deck of 52 playing card (with proper numbers and suits) in the "cards" list. Each cards will be  of the class PlayingCard above<br>
- Allows the user to specify a specific suit (of the 4 - "♠" "♥" "♦" or "♣").  In this case, the program should only populate the deck with the 13 cards of that suit.
- After the cards object is initialized, call the "shuffle_deck()" function (below).<br>
\end{verbatim}

\begin{itemize}
\tightlist
\item
  A "shuffle\_deck()" function that randomly changes the order of cards
  in the deck.
\item
  A "deal\_card(card\_count)" function that removes the first X cards
  from the deck and returns them as a list.

  \begin{itemize}
  \tightlist
  \item
    Make sure this function gives an appropriate response when the deck
    is out of cards.
  \end{itemize}
\end{itemize}

3. You might have to write \texttt{\_\_str\_\_\ or\ \_\_repr\_\_}
methods to display the cards correctly.

Example:

\begin{verbatim}
>>> card1 = PlayingCard("A", "♠")
>>> print(card1)
'A' of ♠

>>> card2 = PlayingCard(15, "♠")
Invalid rank!

>>> card2 = PlayingCard(10, "bunnies")
Invalid suit!

>>> deck1 = Deck()
>>> print(deck1.cards)
['K' of ♠, 'A' of ♥, 6 of ♣, 7 of ♠, 'J' of ♦, 6 of ♠, 'Q' of ♦, 5 of ♣, 10 of ♦, 2 of ♥, 8 of ♣, 8 of ♦, 4 of ♦, 7 of ♦, 3 of ♣, 'K' of ♣, 9 of ♠, 4 of ♥, 10 of ♥, 10 of ♣, 'A' of ♠, 9 of ♥, 7 of ♥, 9 of ♣, 7 of ♣, 5 of ♠, 3 of ♦, 10 of ♠, 'Q' of ♥, 'J' of ♣, 5 of ♥, 'K' of ♥, 'K' of ♦, 2 of ♠, 8 of ♠, 'Q' of ♣, 3 of ♠, 6 of ♥, 6 of ♦, 'A' of ♣, 'A' of ♦, 3 of ♥, 'J' of ♠, 4 of ♣, 5 of ♦, 2 of ♦, 4 of ♠, 2 of ♣, 'Q' of ♠, 'J' of ♥, 8 of ♥, 9 of ♦] 

>>> deck2 = Deck('♠')
>>> deck2.shuffle_deck()
>>> print(deck2.cards)
['A' of ♠, 10 of ♠, 3 of ♠, 7 of ♠, 5 of ♠, 4 of ♠, 8 of ♠, 'J' of ♠, 9 of ♠, 'Q' of ♠, 6 of ♠, 2 of ♠, 'K' of ♠]

>>> deck2.deal_card(7)
['A' of ♠, 10 of ♠, 3 of ♠, 7 of ♠, 5 of ♠, 4 of ♠, 8 of ♠]

>>> deck2.deal_card(8)
Cannot deal 7 cards. The deck only has 6 cards left!
\end{verbatim}

    \begin{Verbatim}[commandchars=\\\{\}]
{\color{incolor}In [{\color{incolor} }]:} \PY{c+c1}{\PYZsh{} YOUR CODE HERE}
\end{Verbatim}


    \begin{Verbatim}[commandchars=\\\{\}]
{\color{incolor}In [{\color{incolor} }]:} \PY{c+c1}{\PYZsh{} Autograde cell \PYZhy{} do not erase/delete}
\end{Verbatim}


    \begin{Verbatim}[commandchars=\\\{\}]
{\color{incolor}In [{\color{incolor} }]:} \PY{c+c1}{\PYZsh{} Autograde cell \PYZhy{} do not erase/delete}
\end{Verbatim}


    \begin{Verbatim}[commandchars=\\\{\}]
{\color{incolor}In [{\color{incolor} }]:} \PY{c+c1}{\PYZsh{} Autograde cell \PYZhy{} do not erase/delete}
\end{Verbatim}


    \begin{Verbatim}[commandchars=\\\{\}]
{\color{incolor}In [{\color{incolor} }]:} \PY{c+c1}{\PYZsh{} Autograde cell \PYZhy{} do not erase/delete}
\end{Verbatim}


    \begin{Verbatim}[commandchars=\\\{\}]
{\color{incolor}In [{\color{incolor} }]:} \PY{c+c1}{\PYZsh{} Autograde cell \PYZhy{} do not erase/delete}
\end{Verbatim}


    \textbf{7-2. Extra Credit:} (2 points) Write a method called
\textbf{war} that deals a card to the player and a card to the dealer
from your deck. Whomever has the highest ranked card wins - print them a
nice message! (2 is the lowest rank and A is the highest). If it is a
tie - print a different message.

    \begin{Verbatim}[commandchars=\\\{\}]
{\color{incolor}In [{\color{incolor} }]:} \PY{c+c1}{\PYZsh{} YOUR CODE HERE}
\end{Verbatim}


    \subsection{7-3. Galton's Box}\label{galtons-box}

The following figure depicts Galton's box, a game in which marbles are
dropped through N rows of pins. In row 0, there is one position a marble
can be in (labeled 0), in row 1, there are two positions (labeled 0 and
1), and so forth. Each time the marble bounces from one row to the next,
there is a 50\% probability it bounces left and a 50\% probability it
bounces right.

Notice that if a marble is in position x of row y, and it bounces left,
it ends up in position x of row y+1. If it bounces right, it ends up in
position x+1.

1. Create a class, \textbf{Marble}, to represent a single Marble that
will drop through Galton's Box. - Include attributes to represent the
position of the marble.\\
- The \texttt{\_\_init\_\_} method should accept a one-character label
for use when printing the Marble.

2. Create a class, \textbf{GaltonBox}, to represent the overall setup.
You should include the following methods:

\begin{itemize}
\tightlist
\item
  \texttt{\_\_init\_\_} - Your initializer should accept the size of the
  box, N.
\item
  \texttt{insert\_marble} - This method should accept a Marble instance
  and sets its position to position 0, row 0.
\item
  \texttt{time\_step} - This method should cause all Marbles in Galton's
  box to bounce to the next row, dropping left or right with equal
  probability. When a marble reaches row N-1 at the bottom of the box,
  it should not move any more. Note that you should simply allow marbles
  to occupy the same position (instead of working out a system to
  prevent a Marble from entering a position if another Marble is already
  there).
\item
  \texttt{\_\_str\_\_\ and\ \_\_repr\_\_} - Include methods to display
  the Marbles currently in the box. To keep things simple, if there are
  multiple Marbles in a given position, you only have to display one of
  the labels.
\end{itemize}

Your classes should mimic the following behavior (except that the
horizontal positions are random):

\begin{verbatim}
>>> m1 = Marble("x")
>>> m2 = Marble("o")
>>> box = GaltonBox(3)
>>> box.insert_marble(m1)
>>> box
x
--
---
>>> box.time_step()
>>> box
-
-x
---
>>> box.insert_marble(m2)
>>> box
o
-x
---
>>> box.time_step()
>>> box
-
o-
-x-
>>> box.time_step()
>>> box
-
--
ox-
\end{verbatim}

    \begin{Verbatim}[commandchars=\\\{\}]
{\color{incolor}In [{\color{incolor} }]:} \PY{c+c1}{\PYZsh{} YOUR CODE HERE}
\end{Verbatim}


    \begin{Verbatim}[commandchars=\\\{\}]
{\color{incolor}In [{\color{incolor} }]:} \PY{c+c1}{\PYZsh{} Autograde cell \PYZhy{} do not erase/delete}
\end{Verbatim}


    \begin{Verbatim}[commandchars=\\\{\}]
{\color{incolor}In [{\color{incolor} }]:} \PY{c+c1}{\PYZsh{} Autograde cell \PYZhy{} do not erase/delete}
\end{Verbatim}


    \begin{Verbatim}[commandchars=\\\{\}]
{\color{incolor}In [{\color{incolor} }]:} \PY{c+c1}{\PYZsh{} Autograde cell \PYZhy{} do not erase/delete}
\end{Verbatim}


    \begin{Verbatim}[commandchars=\\\{\}]
{\color{incolor}In [{\color{incolor} }]:} \PY{c+c1}{\PYZsh{} Autograde cell \PYZhy{} do not erase/delete}
\end{Verbatim}


    \textbf{7-3. Extra Credit: } (2 points) Once your code is working, write
a script to create a box with 20 rows, insert a few dozen Marbles, and
repeatedly call time\_step() until all Marbles are at the bottom. Now
adapt the following code to display a histogram of the final Marble
positions. What does the shape of the distribution look like?

    \begin{Verbatim}[commandchars=\\\{\}]
{\color{incolor}In [{\color{incolor} }]:} \PY{c+c1}{\PYZsh{} Sample code below \PYZhy{} you will have to adjust it to work for your code.}
        
        \PY{o}{\PYZpc{}}\PY{k}{matplotlib} inline
        \PY{k+kn}{import} \PY{n+nn}{matplotlib}\PY{n+nn}{.}\PY{n+nn}{pyplot} \PY{k}{as} \PY{n+nn}{plt}
        \PY{k+kn}{import} \PY{n+nn}{numpy} \PY{k}{as} \PY{n+nn}{np}
        
        \PY{n}{x\PYZus{}positions} \PY{o}{=} \PY{p}{(}\PY{l+m+mi}{2}\PY{p}{,}\PY{l+m+mi}{3}\PY{p}{,}\PY{l+m+mi}{4}\PY{p}{,}\PY{l+m+mi}{6}\PY{p}{,}\PY{l+m+mi}{7}\PY{p}{,}\PY{l+m+mi}{4}\PY{p}{,}\PY{l+m+mi}{3}\PY{p}{,}\PY{l+m+mi}{2}\PY{p}{,}\PY{l+m+mi}{3}\PY{p}{,}\PY{l+m+mi}{1}\PY{p}{)}
        \PY{n}{cutoffs} \PY{o}{=} \PY{n}{np}\PY{o}{.}\PY{n}{arange}\PY{p}{(}\PY{n+nb}{min}\PY{p}{(}\PY{n}{x\PYZus{}positions}\PY{p}{)} \PY{o}{\PYZhy{}} \PY{o}{.}\PY{l+m+mi}{5}\PY{p}{,} \PY{n+nb}{max}\PY{p}{(}\PY{n}{x\PYZus{}positions}\PY{p}{)}\PY{o}{+}\PY{o}{.}\PY{l+m+mi}{5}\PY{p}{)}
        \PY{n}{plt}\PY{o}{.}\PY{n}{hist}\PY{p}{(}\PY{n}{x\PYZus{}positions}\PY{p}{,} \PY{n}{bins} \PY{o}{=} \PY{n}{cutoffs}\PY{p}{)}
\end{Verbatim}


    \begin{Verbatim}[commandchars=\\\{\}]
{\color{incolor}In [{\color{incolor} }]:} \PY{c+c1}{\PYZsh{} YOUR CODE HERE}
\end{Verbatim}


    YOUR ANSWER HERE

    \subsection{7-4. Sorting Marbles}\label{sorting-marbles}

In a particular board game, there are N spaces in a row, numbered 0
through N - 1 from left to right. There are also N marbles, numbered 0
through N - 1, initially placed in some arbitrary order. After that,
there are two moves available:

\begin{itemize}
\tightlist
\item
  Switch: Switch the marbles in positions 0 and 1.
\item
  Rotate: Move the marble in position 0 to position N - 1, and move all
  other marbles one space to the left (one index lower).
\end{itemize}

The objective is to arrange the marbles in order, with each marble i in
position i.

1. Write a class, \textbf{MarblesBoard}, to represent the game above.
The class should be initialized with a particular sequence of Marbles.\\
- Write an \texttt{\_\_init\_\_} function that takes a starting sequence
set of marbles (the number of each marble listed in the positions from 0
to N - 1). (Notice the sequence is a set) - Next, write switch() and
rotate() methods to simulate the player's moves as described above. -
Write a method, is\_solved(), that returns True if the marbles are in
the correct order. - Additionally, you may want to write
\texttt{\_\_str\_\_\ and\ \_\_repr\_\_} methods that display the current
state of the board.

Your class should behave like the following example:

\begin{verbatim}
>>> board = MarblesBoard((3,6,7,4,1,0,8,2,5)) 
>>> board 
3 6 7 4 1 0 8 2 5 
>>> board.switch() 
>>> board 
6 3 7 4 1 0 8 2 5 
>>> board.rotate() 
>>> board 
3 7 4 1 0 8 2 5 6 
>>> board.switch() 
>>> board 
7 3 4 1 0 8 2 5 6
\end{verbatim}

2. Write a second class, \textbf{Solver}, that actually plays the
MarblesGame. - Your class will take a MarblesBoard in its initializer. -
Write a solve() method that repeatedly calls the switch() and rotate()
methods of the given MarblesBoard until the game is solved. You should
print the state of the board after each move. Additionally, print out
the total number of moves at the end.

You are to come up with your own algorithm for solving the marbles game.
Before you write your solve() method, you may want to practice solving
some small versions of the marbles game yourself.

Here is an example:

\begin{verbatim}
>>> board2 = MarblesBoard((1,3,0,2))
>>> solver = Solver(board2)
>>> solver.solve()
1 3 0 2 
3 0 2 1 
0 2 1 3 
2 1 3 0 
1 2 3 0 
2 3 0 1 
3 0 1 2 
0 1 2 3 
total steps: 7
\end{verbatim}

Your Solver does not need to follow the fastest possible running time
but you should strive to make it reasonably efficient.

You may be interested to know that your program is a variation of a
well-known sorting algorithm called bubble sort. Bubble sort would
normally be used on a list of items, not on a rotating track, but
adapting your algorithm to this setting would be easy.

    \begin{Verbatim}[commandchars=\\\{\}]
{\color{incolor}In [{\color{incolor} }]:} \PY{c+c1}{\PYZsh{} YOUR CODE HERE}
\end{Verbatim}


    \begin{Verbatim}[commandchars=\\\{\}]
{\color{incolor}In [{\color{incolor} }]:} \PY{c+c1}{\PYZsh{} Autograde cell \PYZhy{} do not erase/delete}
\end{Verbatim}


    \begin{Verbatim}[commandchars=\\\{\}]
{\color{incolor}In [{\color{incolor} }]:} \PY{c+c1}{\PYZsh{} Autograde cell \PYZhy{} do not erase/delete}
\end{Verbatim}


    \begin{Verbatim}[commandchars=\\\{\}]
{\color{incolor}In [{\color{incolor} }]:} \PY{c+c1}{\PYZsh{} Autograde cell \PYZhy{} do not erase/delete}
\end{Verbatim}


    \begin{Verbatim}[commandchars=\\\{\}]
{\color{incolor}In [{\color{incolor} }]:} \PY{c+c1}{\PYZsh{} Autograde cell \PYZhy{} do not erase/delete}
\end{Verbatim}


    \begin{Verbatim}[commandchars=\\\{\}]
{\color{incolor}In [{\color{incolor} }]:} \PY{c+c1}{\PYZsh{} Autograde cell \PYZhy{} do not erase/delete}
\end{Verbatim}


    \textbf{7-4. Extra Credit:} (2 points) Below, explain what the big-O
running time of your algorithm is.

    YOUR ANSWER HERE


    % Add a bibliography block to the postdoc
    
    
    
    \end{document}
